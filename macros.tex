% ------------------------------------------------------------------------------
% ------------------------------------------------------------------------------
%
%
\newcommand*{\todo}[1]{[\textbf{\color{red} Todo:} \textit{#1}]}
\newcommand*{\note}[1]{[\textbf{\color{blue} Note:} \textit{#1}]}
\newenvironment{mynote}{\begin{quote}\color{gray}\textbf{\underline{Note:} }}{\end{quote}}
%
%
%
% Math modes can be tested for: \ifmmode is true in display and non-display math mode, and \ifinner is true in non-display mode, but not in display mode.
%\edef\modedef{This macro was defined in \ifvmode vertical\else \ifmmode math \else horizontal\fi\fi' mode}
%
%
\renewcommand{\subset}{\subseteq}
%
%
% brackets
\newcommand{\lcb}{\left\lbrace} % '{' Left Curly Bracket
\newcommand{\rcb}{\right\rbrace} % '}' Right Curly Bracket
\newcommand{\cb}[1]{\lcb #1 \rcb} % Curly Brackets
\newcommand{\cbOf}[1]{\mathopen{}\lcb #1 \rcb\mathclose{}} % Curly Brackets
\newcommand{\lab}{\left[} %'[' Left square/Angular Bracket
\newcommand{\rab}{\right]} %']' Right square/Angular Bracket
\newcommand{\ab}[1]{\lab #1 \rab} % Brackets
\newcommand{\abOf}[1]{\!\ab{#1}} % Brackets
\newcommand{\lb}{\left(} %'(' Left (round) Bracket
\newcommand{\rb}{\right)} %')' Right (round) Bracket
\newcommand{\br}[1]{\lb #1 \rb} % Brackets
\newcommand{\brOf}[1]{\!\br{#1}} % Brackets
\newcommand{\abs}[1]{\left| #1 \right|} % Brackets
%
% MathOperators
\newcommand*{\E}{\mathbb{E}} % expectation
\newcommand*{\ent}{\mathsf{Ent}} % entropy
\newcommand*{\V}{\mathbb{V}} % expectation
\newcommand*{\cov}{\mathbb{COV}} % expectation
\let\Pr\relax% Set equal to \relax so that LaTeX thinks it's not defined
\newcommand*{\Pr}{\mathbb{P}} % probability
\newcommand*{\powset}{\mathfrak{P}} % power set
% *Of (inserts brackets, mathopen/-close used for correct spacing)
\newcommand{\sizedMid}[2]{#1 \, \kern-\nulldelimiterspace\mathopen{}\left| \vphantom{#1}\,#2\right.\mathclose{}\kern-\nulldelimiterspace}
\newcommand{\EOf}[1]{\E\abOf{#1}}
\newcommand{\Eof}[1]{\E[#1]}
\newcommand{\entOf}[1]{\ent\brOf{#1}}
\newcommand{\entof}[1]{\ent(#1)}
\newcommand{\Vof}[1]{\V[#1]}
\newcommand{\VOf}[1]{\V\abOf{#1}}
\newcommand{\covof}[1]{\cov[#1]}
\newcommand{\ECondOf}[2]{\EOf{\sizedMid{#1}{#2}}}
\newcommand{\ECondof}[2]{\E[#1\mid #2]}
\newcommand{\PrOf}[1]{\Pr\mathopen{}\lb #1 \rb\mathclose{}}
\newcommand{\Prof}[1]{\Pr(#1)}
\newcommand{\PrCondOf}[2]{\PrOf{\sizedMid{#1}{#2}}}
\newcommand{\PrCondof}[2]{\Pr(#1\mid #2)}
\newcommand{\powsetOf}[1]{\powset\mathopen{}\lb #1 \rb\mathclose{}}
\newcommand{\eventAnd}{,\,} %eventAnd
\newcommand{\setByEle}[2]{\cb{\sizedMid{#1}{#2}}}
\newcommand{\setByEleInText}[2]{\{#1 \mid #2\}}
\DeclareMathOperator{\diam}{\mathsf{diam}}
\DeclareMathOperator{\card}{\mathsf{card}}
\DeclareMathOperator{\ball}{\mathrm{B}}
\newcommand{\id}{\mathrm{id}}
\newcommand{\restrict}[1]{_{\mkern 1mu \vrule height 2ex\mkern2mu {#1}}}
%
\providecommand\given{} % so it exists
\newcommand\SetSymbol[1][]{
	\nonscript\,#1\vert \allowbreak \nonscript\,\mathopen{}}
\DeclarePairedDelimiterX\Set[1]{\lbrace}{\rbrace}%
{ \renewcommand\given{\SetSymbol[\delimsize]} #1 }
%
%\newcommand\myEx[1][]{\nonscript\,#1\vert \allowbreak \nonscript\,\mathopen{}}
%\DeclarePairedDelimiterX{\myexpectarg}[1]{\lbrace}{\rbrace}{\renewcommand\given{\myEx[\delimsize]} #1 }
%
\newcommand{\Ex}{\E\expectarg}
\DeclarePairedDelimiterX{\expectarg}[1]{[}{]}{%
	\ifnum\currentgrouptype=16 \else\begingroup\fi
	\activatebar#1
	\ifnum\currentgrouptype=16 \else\endgroup\fi
}
\newcommand{\innermid}{\nonscript\;\delimsize\vert\nonscript\;}
\newcommand{\activatebar}{%
	\begingroup\lccode`\~=`\|
	\lowercase{\endgroup\let~}\innermid 
	\mathcode`|=\string"8000
}
%
% cardinality of a set: |A|
\newcommand{\cardOf}[1]{\left\vert{#1}\right\vert}
\newcommand{\cardof}[1]{\vert{#1}\vert}
%
% image of a map
\newcommand*{\im}{\mathrm{im}}
% 
\newcommand*{\mc}[1]{\mathcal{#1}}
\newcommand*{\mb}[1]{\mathbb{#1}}
\newcommand*{\mr}[1]{\mathrm{#1}}
\newcommand*{\ms}[1]{\mathsf{#1}}
\newcommand*{\mo}[1]{\mathbf{#1}}
\newcommand*{\mf}[1]{\mathfrak{#1}}
%
% natural numbers
\newcommand{\N}{\mathbb{N}}
\newcommand{\Nn}{\mathbb{N}_0}
% real numbers
\newcommand{\R}{\mathbb{R}}
\newcommand{\Rp}{\R_+} % [0, \infty)
\newcommand{\Ra}{\bar\R} % extended reals
% rational numbers
\newcommand{\Q}{\mathbb{Q}}
% rational numbers
\newcommand{\Z}{\mathbb{Z}}
%
% borel
\DeclareMathOperator{\borel}{\mathcal{B}}
\newcommand{\borelof}[1]{\borel(#1)}
%
% C
\newcommand{\C}[2]{\mathcal{C}^{#1}\br{#2}}
% 
% derivative
\newcommand{\derive}[1]{#1^\prime}
%
\newcommand{\transpose}{\!^\top\!}
\newcommand{\tr}{\transpose}
%
\newcommand{\pr}{^\prime}
\newcommand{\prr}{^{\prime\prime}}
\newcommand{\prrr}{^{\prime\prime\prime}}
% 
% [0, 1]
\newcommand{\normIntervall}{\lab 0, 1 \rab} %
%
% Integral
%\newcommand{\intOf}[4]{\int_{#1}^{#2} \! #3 \mathrm{d}#4} %
\def\integral from #1to #2of #3by #4;{\int_{#1}^{#2} \! #3 \mathrm{d}#4} %
\def\integralMeasure in #1of #2by #3of #4;{\int_{#1} \! #2{#4} #3{\mathrm{d}#4}} %
%
% Functiondefintion
%\newcommand{\mapping}[3]{#1 \colon #2 \rightarrow #3}
\def\mapping #1from #2to #3;{#1 \colon #2 \rightarrow #3}
\def\mappingDef #1from #2to #3maps #4to #5;{#1 \colon #2 \rightarrow #3,\ #4 \mapsto #5}
%
% Sequence (Folge)
\def\seq #1by #2;{\br{#1}_{#2\in\N}}
\def\seqInText #1by #2;{(#1)_{#2\in\N}}
%
%
\newcommand{\innerProduct}[2]{\left\langle#1\,,\, #2\right\rangle}
\newcommand{\ip}[2]{\innerProduct{#1}{#2}}
%
%
\newcommand{\lebesgue}{\mathcal{L}}
\newcommand{\lebesguePow}[1]{\lebesgue^{#1}}
\newcommand{\lebesgueOf}[1]{\lebesgue\brOf{#1}}
%
%
\newcommand{\invert}[1]{#1^{-1}}
%
\newcommand{\sgn}{\mathsf{sgn}} % signum
\newcommand{\supp}{\overline{\mathsf{supp}}} % support
\newcommand{\Cut}{\mathsf{Cut}} % cut locus
\newcommand{\median}{\mathsf{median}} % median
\newcommand{\Exp}{\mathsf{Exp}}
\newcommand{\Log}{\mathsf{Log}}
%
%
\newcommand{\dl}{\mathrm{d}}
%
\newcommand\independent{\protect\mathpalette{\protect\independenT}{\perp}}
\def\independenT#1#2{\mathrel{\rlap{$#1#2$}\mkern2mu{#1#2}}}
%
%
\def\converges for #1to #2;{\xrightarrow{#1} #2}
\def\convergesAlmostSurely for #1to #2;{\xrightarrow{#1}_{\mathsf{fs}} #2}
\def\convergesInProbability for #1to #2;{\xrightarrow{#1}_{\mathsf{p}} #2}
\def\convergesInL #1for #2to #3;{\xrightarrow{#2}_{\lebesguePow{#1}} #3}
\newcommand{\as}{\mathsf{a.s.}}
%
% Indicator
\newcommand{\ind}{\mathds{1}}% indicator function
\newcommand{\indOf}[1]{\ind_{\!#1}}% 
\newcommand{\indOfOf}[2]{\ind_{\!#1}\!\brOf{#2}}% 
\newcommand{\indOfEvent}[1]{\indOf{\cbOf{#1}}}% 
\newcommand{\indOfEventOf}[2]{\indOf{\cbOf{#1}}\!\brOf{#2}}% 
%
% complement of a set
%\newcommand{\complOf}[1]{#1^\complement}% 
%\newcommand{\complOf}[1]{\inSpace\setminus#1}% 
\newcommand{\compl}{^\mathsf{c}}% 
%
% Landau symbols
\DeclareMathOperator{\landauO}{\mathcal{O}}%
\newcommand{\landauOOf}[1]{\landauO\brOf{#1}}% 
\DeclareMathOperator{\landauOmega}{\Omega}%
\newcommand{\landauOmegaOf}[1]{\landauOmega\brOf{#1}}% 
\DeclareMathOperator{\landauTheta}{\Theta}%
\newcommand{\landauThetaOf}[1]{\landauTheta\brOf{#1}}% 
%
% norm
\newcommand{\norm}{\left\Vert \cdot \right\Vert}
\newcommand{\normof}[1]{\Vert #1 \Vert}
\newcommand{\normOf}[1]{\left\Vert #1 \right\Vert}
%
% ditributions
\newcommand{\unif}[1]{\mathsf{Unif}\brOf{#1}}
%
%
\newcommand{\equationFullstop}{\, .}
\newcommand{\eqfs}{\equationFullstop}
\newcommand{\equationComma}{\, ,}
\newcommand{\eqcm}{\equationComma}
\newcommand{\eqand}{\qquad\text{and}\qquad}
%
%
\newcommand{\sigmaAlg}{\mathrm{\sigma}}
\newcommand{\sigmaAlgof}[1]{\sigmaAlg(#1)}
\newcommand{\sigmaAlgOf}[1]{\sigmaAlg\brOf{#1}}
%
%
\newcommand{\euler}{\mathrm{e}}
%
%
%
\newcommand{\T}{^{T}}
%
%
%\newcommand*{\argmin}{\arg\min}
\DeclareMathOperator*{\argmin}{arg\,min}
\DeclareMathOperator*{\argmax}{arg\,max}
\DeclareMathOperator*{\outerlim}{lim\, \overline{sup}}
\DeclareMathOperator*{\innerlim}{innerlim}
%
%
%
%
\newcommand*{\evmin}{\lambda^{\mathsf{min}}}
\newcommand*{\evmax}{\lambda^{\mathsf{max}}}
\newcommand{\entInt}[1]{\mathrm J[#1]}
\def\MS{\mathcal{Q}}%the metric space
\def\epi{\mathsf{epi}}%
\def\aec{\mathsf{aec}}%
\def\alsec{\mathsf{alsec}}%
\def\aecp{\mathsf{aec}_P}%
\def\Hausdorff{\mathsf{H}}%
\def\Haus{\mathsf{H}}%
\def\ubs{\mathsf{ubs}}%
\newcommand{\epiconv}[1]{\xrightarrow{#1\to\infty}_\epi}%
\newcommand{\hausconv}[1]{\xrightarrow{#1\to\infty}_\Hausdorff}%
\newcommand{\ubsconv}[1]{\xrightarrow{#1\to\infty}_\ubs}%
\newcommand{\ol}[2]{\overline{#1,\!#2}}
\newcommand{\sol}[2]{\overline{#1,#2}}
%
%
%
%
%
%
%
%
%
% ------------------------------------------------------------------------------
% ------------------------------------------------------------------------------
